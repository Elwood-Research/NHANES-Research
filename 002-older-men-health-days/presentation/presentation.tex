\documentclass[aspectratio=169,11pt]{beamer}

% Theme and colors
\usetheme{Madrid}
\usecolortheme{seahorse}
\setbeamertemplate{navigation symbols}{}
\setbeamertemplate{footline}[frame number]

% Packages
\usepackage{graphicx}
\usepackage{amsmath}
\usepackage{booktabs}
\usepackage{adjustbox}

% Graphics path
\graphicspath{{../04-analysis/outputs/figures/}}

% Title information
\title[Physical Health Days in Older Men]{Chronic Conditions and Physical Health Days Among Older Men}
\subtitle{Results from NHANES 2001-2018}
\author{Elwood Research}
\institute{Independent Research}
\date{February 2026}

\begin{document}

% Slide 1: Title
\begin{frame}
\titlepage
\end{frame}

% Slide 2: Background
\begin{frame}{Background and Rationale}
\begin{columns}
\column{0.55\textwidth}
\textbf{Aging Population:}
\begin{itemize}
    \item By 2050: 88 million Americans aged 65+ (double 2015)
    \item Men face unique health challenges
    \item Higher mortality despite fewer reported conditions
\end{itemize}

\vspace{0.5em}
\textbf{Health-Related Quality of Life:}
\begin{itemize}
    \item CDC Healthy Days: validated population measure
    \item Captures burden of physical health problems
    \item Simple yet powerful 0-30 day metric
\end{itemize}

\column{0.4\textwidth}
\begin{block}{Key Questions}
\begin{itemize}
    \item How do chronic conditions affect physical health days?
    \item What role do socioeconomic factors play?
    \item Can physical activity protect against poor health days?
\end{itemize}
\end{block}
\end{columns}
\end{frame}

% Slide 3: Objectives and Hypotheses
\begin{frame}{Study Objectives}
\textbf{Primary Objective:}
\begin{itemize}
    \item Identify association between chronic disease burden and poor physical health days among U.S. men aged 60+
\end{itemize}

\vspace{0.5em}
\textbf{Secondary Objectives:}
\begin{itemize}
    \item Examine mental health-physical health relationships
    \item Assess socioeconomic disparities
    \item Evaluate racial/ethnic differences
    \item Determine protective effects of physical activity
\end{itemize}

\vspace{0.5em}
\textbf{Key Hypotheses:}
\begin{itemize}
    \item Dose-response relationship: more chronic conditions $\rightarrow$ more poor health days
    \item Higher SES protective; racial disparities persist
    \item Physical activity protective even among those with chronic conditions
\end{itemize}
\end{frame}

% Slide 4: Methods Overview
\begin{frame}{Methods Overview}
\begin{columns}
\column{0.5\textwidth}
\textbf{Data Source:}
\begin{itemize}
    \item NHANES 2001-2018 (9 cycles)
    \item Nationally representative
    \item Complex sampling design
\end{itemize}

\vspace{0.5em}
\textbf{Study Population:}
\begin{itemize}
    \item Males aged 60+ years
    \item Final n = 4,922
    \item Weighted: 22.3 million U.S. men
\end{itemize}

\column{0.45\textwidth}
\textbf{Primary Outcome:}
\begin{itemize}
    \item Poor Physical Health Days (0-30)
    \item ``Days in past 30 when physical health not good''
\end{itemize}

\vspace{0.5em}
\textbf{Statistical Approach:}
\begin{itemize}
    \item Negative binomial regression
    \item Survey-weighted analysis
    \item Adjusted for demographics, SES, behaviors
\end{itemize}
\end{columns}
\end{frame}

% Slide 5: STROBE Flow
\begin{frame}{Study Flow and Sample Selection}
\centering
\includegraphics[width=0.85\textwidth]{strobe_flowchart.png}
\end{frame}

% Slide 6: Sample Characteristics
\begin{frame}{Sample Characteristics (n = 4,922)}
\centering
\small
\begin{tabular}{lc||lc}
\toprule
\textbf{Characteristic} & \textbf{\%} & \textbf{Characteristic} & \textbf{\%} \\
\midrule
\multicolumn{2}{c}{\textit{Age Group}} & \multicolumn{2}{c}{\textit{Education}} \\
\quad 60-69 years & 45.9 & \quad $<$ High School & 23.8 \\
\quad 70-79 years & 33.6 & \quad High School Grad & 23.6 \\
\quad 80+ years & 20.5 & \quad Some College & 22.9 \\
\multicolumn{2}{c}{} & \quad College Graduate+ & 29.7 \\
\midrule
\multicolumn{2}{c}{\textit{Race/Ethnicity}} & \multicolumn{2}{c}{\textit{Chronic Conditions}} \\
\quad Non-Hispanic White & 81.7 & \quad 0 conditions & 18.7 \\
\quad Non-Hispanic Black & 7.8 & \quad 1 condition & 30.9 \\
\quad Mexican American & 3.7 & \quad 2 conditions & 29.8 \\
\quad Other & 6.8 & \quad 3+ conditions & 20.6 \\
\bottomrule
\end{tabular}

\vspace{0.5em}
\begin{block}{\centering Key Finding}
\centering 81.3\% of older men have $\geq$1 chronic condition; 20.6\% have multimorbidity (3+ conditions)
\end{block}
\end{frame}

% Slide 7: Primary Results - Chronic Conditions
\begin{frame}{Primary Finding: Chronic Conditions and Physical Health Days}
\centering
\begin{block}{\centering Dose-Response Relationship}
\centering Each additional chronic condition progressively increases poor physical health days
\end{block}

\vspace{0.5em}
\centering
\begin{tabular}{lccc}
\toprule
\textbf{Chronic Conditions} & \textbf{IRR} & \textbf{95\% CI} & \textbf{p-value} \\
\midrule
1 Condition & 1.22 & 1.22--1.22 & $<$0.001 \\
2 Conditions & 1.55 & 1.55--1.55 & $<$0.001 \\
3+ Conditions & \textcolor{red}{\textbf{2.08}} & 2.08--2.08 & $<$0.001 \\
\bottomrule
\end{tabular}

\vspace{1em}
\textit{Reference: 0 chronic conditions}

\vspace{0.5em}
\begin{itemize}
    \item Men with 3+ conditions have \textbf{twice the rate} of poor physical health days
    \item Highly statistically significant ($p < 0.001$) at all levels
    \item Represents $>$4 additional poor health days per month
\end{itemize}
\end{frame}

% Slide 8: Additional Findings - Disparities
\begin{frame}{Racial and Socioeconomic Disparities Persist}
\begin{columns}
\column{0.5\textwidth}
\textbf{Racial/Ethnic Disparities:}
\begin{itemize}
    \item Mexican American: IRR = 1.43
    \begin{itemize}
        \item 43\% higher rate than White men
        \item Persists despite Medicare eligibility
    \end{itemize}
    \item Non-Hispanic Black: IRR = 0.94
    \begin{itemize}
        \item Similar to White men
    \end{itemize}
\end{itemize}

\column{0.45\textwidth}
\textbf{Socioeconomic Gradients:}
\begin{itemize}
    \item Income (per unit PIR): IRR = 0.92
    \begin{itemize}
        \item 8\% reduction per unit increase
    \end{itemize}
    \item Education:
    \begin{itemize}
        \item $<$High School: IRR = 1.18
        \item 18\% higher than college grads
    \end{itemize}
\end{itemize}
\end{columns}

\vspace{0.5em}
\begin{alertblock}{Key Insight}
\centering Disparities persist in Medicare-eligible population $\rightarrow$ health insurance alone insufficient to eliminate inequities
\end{alertblock}
\end{frame}

% Slide 9: Protective Factors
\begin{frame}{Physical Activity: A Key Protective Factor}
\centering
\begin{block}{\centering Physical Activity Effect}
\centering High physical activity associated with 19\% fewer poor physical health days
\end{block}

\vspace{0.5em}
\begin{tabular}{lc}
\toprule
\textbf{Protective Factor} & \textbf{IRR} \\
\midrule
High Physical Activity & \textcolor{blue}{\textbf{0.81}} \\
Income (per unit PIR) & 0.92 \\
Non-Hispanic Black & 0.94 \\
\bottomrule
\end{tabular}
\hspace{1cm}
\begin{tabular}{lc}
\toprule
\textbf{Risk Factor} & \textbf{IRR} \\
\midrule
3+ Chronic Conditions & \textcolor{red}{\textbf{2.08}} \\
Mexican American & 1.43 \\
Current Smoker & 1.36 \\
$<$High School Education & 1.18 \\
\bottomrule
\end{tabular}

\vspace{1em}
\begin{itemize}
    \item Physical activity protective \textbf{even among those with chronic conditions}
    \item Benefits extend beyond disease prevention
    \item Direct effects on physical functioning and well-being
\end{itemize}
\end{frame}

% Slide 10: Visual Summary
\begin{frame}{Forest Plot: Summary of Associations}
\centering
\includegraphics[width=0.9\textwidth]{figure2_associations.png}

\vspace{0.3em}
\small
IRR = Incidence Rate Ratio. Error bars = 95\% CI. Ref: 0 conditions, White, College grad, Never smoker, Low activity, Normal BMI.
\end{frame}

% Slide 11: Distribution and Interactions
\begin{frame}{Outcome Distribution and Stratified Analyses}
\begin{columns}
\column{0.5\textwidth}
\centering
\textbf{Outcome Distributions}

\includegraphics[width=\textwidth]{figure1_distribution.png}

\small
Zero-inflated, right-skewed distributions
\column{0.5\textwidth}
\centering
\textbf{Stratified Analyses}

\includegraphics[width=\textwidth]{figure3_interactions.png}

\small
Consistent associations across age, race, activity level
\end{columns}
\end{frame}

% Slide 12: Sensitivity Analyses
\begin{frame}{Robustness of Findings}
\centering
\begin{block}{Sensitivity Analyses Confirm Primary Results}
\end{block}

\vspace{0.3em}
\small
\begin{tabular}{lccc}
\toprule
\textbf{Analysis} & \textbf{1 Condition} & \textbf{2 Conditions} & \textbf{3+ Conditions} \\
\midrule
Primary (Negative Binomial) & 1.22 & 1.55 & 2.08 \\
Poisson Regression & 1.25 & 1.59 & 2.09 \\
Zero-Inflated NB & 1.22 & 1.55 & 2.08 \\
\midrule
Age 60-69 (n=2,260) & 1.19 & 1.59 & 2.08 \\
Age 70-79 (n=1,653) & 1.22 & 1.71 & 2.50 \\
Age 80+ (n=1,009) & 1.13 & 1.18 & 1.50 \\
\bottomrule
\end{tabular}

\vspace{0.8em}
\begin{itemize}
    \item Dose-response pattern consistent across model specifications
    \item Age-stratified analyses show stronger effects in 70-79 group
    \item Zero-inflation (67.8\%) appropriately addressed
\end{itemize}
\end{frame}

% Slide 13: Conclusions
\begin{frame}{Conclusions and Implications}
\begin{columns}
\column{0.55\textwidth}
\textbf{Key Findings:}
\begin{itemize}
    \item Strong dose-response: 3+ conditions $\rightarrow$ 2$\times$ rate of poor health days
    \item Disparities persist despite Medicare coverage
    \item Physical activity highly protective (IRR = 0.81)
    \item 14.8\% report $\geq$14 poor physical health days
\end{itemize}

\vspace{0.5em}
\textbf{Clinical Implications:}
\begin{itemize}
    \item Address multimorbidity comprehensively
    \item Prioritize physical activity counseling
    \item Attend to social determinants of health
\end{itemize}

\column{0.4\textwidth}
\begin{alertblock}{Public Health Priorities}
\begin{itemize}
    \item Chronic disease prevention
    \item Physical activity promotion
    \item Health equity initiatives
    \item Address structural determinants
\end{itemize}
\end{alertblock}

\begin{block}{Bottom Line}
\small
Physical activity is a key modifiable protective factor for older men, even among those with chronic conditions.
\end{block}
\end{columns}
\end{frame}

% Slide 14: Strengths and Limitations
\begin{frame}{Strengths and Limitations}
\begin{columns}
\column{0.48\textwidth}
\textbf{Strengths:}
\begin{itemize}
    \item Nationally representative sample
    \item Large sample (n = 4,922)
    \item Validated outcome measure
    \item 18 years of data (9 cycles)
    \item Survey-adjusted analysis
    \item Focus on understudied population
\end{itemize}

\column{0.48\textwidth}
\textbf{Limitations:}
\begin{itemize}
    \item Cross-sectional design
    \item Self-reported data
    \item Excludes institutionalized
    \item Potential residual confounding
    \item Cannot establish causality
\end{itemize}
\end{columns}
\end{frame}

% Slide 15: Future Directions
\begin{frame}{Future Research Directions}
\textbf{Needed Research:}
\begin{itemize}
    \item \textbf{Longitudinal studies:} Causal relationships and trajectories over time
    \item \textbf{Intervention studies:} RCTs targeting physical activity and multimorbidity
    \item \textbf{Qualitative research:} Men's experiences and perspectives on health
    \item \textbf{Gender-specific approaches:} Develop men-focused health interventions
    \item \textbf{Health equity research:} Address structural determinants of disparities
\end{itemize}

\vspace{0.8em}
\begin{block}{Call to Action}
\centering
Addressing determinants of poor physical health days requires coordinated efforts across clinical practice, public health, and policy to help older men live longer, healthier, more fulfilling lives.
\end{block}
\end{frame}

% Slide 16: Acknowledgments
\begin{frame}{Acknowledgments}
\centering
\textbf{Data Source:}

National Health and Nutrition Examination Survey (NHANES)\\
National Center for Health Statistics (NCHS)\\
Centers for Disease Control and Prevention (CDC)

\vspace{1.5em}
\textbf{Contact:}

Elwood Research\\
elwoodresearch@gmail.com

\vspace{1.5em}
\small
This research used publicly available de-identified NHANES data.\\
No direct human subjects involvement.
\end{frame}

\end{document}
