\documentclass{beamer}
\usetheme{Madrid}
\usecolortheme{default}
\usepackage{booktabs}

\title[Diet and Periodontitis]{Association Between Dietary Nutrient Intake and Periodontitis: A Cross-Sectional Study of NHANES 2009-2014}
\author{Elwood Research}
\institute[]{elwoodresearch@gmail.com}
\date{\today}

\begin{document}

\begin{frame}
  \titlepage
\end{frame}

\begin{frame}{Background}
  \begin{itemize}
    \item \textbf{Periodontitis}: Chronic inflammatory disease destroying tooth-supporting tissues.
    \item \textbf{Prevalence}: Affects nearly half of U.S. adults aged 30+.
    \item \textbf{Risk Factors}: 
      \begin{itemize}
        \item Established: Smoking, Diabetes, Poor Oral Hygiene.
        \item Emerging: Diet and Nutrition.
      \end{itemize}
    \item \textbf{Dietary Hypothesis}:
      \begin{itemize}
        \item Pro-inflammatory: Added sugars.
        \item Anti-inflammatory: Fiber, Antioxidants (Vit C), Bone-supporting nutrients (Calcium).
      \end{itemize}
  \end{itemize}
\end{frame}

\begin{frame}{Objectives}
  To investigate the independent association between specific dietary nutrient intakes and the prevalence of moderate-to-severe periodontitis in U.S. adults.
  \vspace{1em}
  \begin{block}{Key Nutrients Analyzed}
    \begin{itemize}
      \item Total Sugars (g/day)
      \item Dietary Fiber (g/day)
      \item Vitamin C (mg/day)
      \item Calcium (mg/day)
    \end{itemize}
  \end{block}
\end{frame}

\begin{frame}{Methods: Design and Population}
  \begin{itemize}
    \item \textbf{Data Source}: NHANES 2009-2014 (3 cycles).
    \item \textbf{Design}: Cross-sectional, nationally representative.
    \item \textbf{Inclusion Criteria}:
      \begin{itemize}
        \item Age $\ge$ 30 years.
        \item Completed full-mouth periodontal examination.
        \item Valid Day 1 dietary recall data.
      \end{itemize}
    \item \textbf{Exclusions}: Edentulous, pregnant, missing covariates.
    \item \textbf{Final Sample Size}: N = 8,006.
  \end{itemize}
\end{frame}

\begin{frame}{Methods: Variables}
  \begin{itemize}
    \item \textbf{Outcome}: Periodontitis (CDC/AAP Definitions)
      \begin{itemize}
        \item Dichotomized: \textbf{Moderate/Severe} vs. \textbf{None/Mild}.
      \end{itemize}
    \item \textbf{Exposures}: Continuous dietary intake from 24h recall.
    \item \textbf{Covariates}:
      \begin{itemize}
        \item \textbf{Demographics}: Age, Sex, Race/Ethnicity, Education, PIR.
        \item \textbf{Behaviors}: Smoking, Alcohol, Physical Activity, Flossing.
        \item \textbf{Clinical}: BMI, Diabetes, Number of Teeth.
        \item \textbf{Dietary}: Total Energy Intake (kcal).
      \end{itemize}
  \end{itemize}
\end{frame}

\begin{frame}{Methods: Analysis}
  \begin{itemize}
    \item \textbf{Software}: Python (pandas, statsmodels).
    \item \textbf{Weighting}: 6-year exam weights (`WTMEC6YR`) to account for complex survey design.
    \item \textbf{Statistical Tests}:
      \begin{itemize}
        \item Descriptive: Weighted means and frequencies.
        \item Modeling: Multivariable Logistic Regression.
      \end{itemize}
    \item \textbf{Models}:
      \begin{itemize}
        \item Model 1: Unadjusted.
        \item Model 2: Adjusted for Demographics + Energy.
        \item Model 3: Fully Adjusted (Health behaviors, SES, Clinical).
      \end{itemize}
  \end{itemize}
\end{frame}

\begin{frame}{Results: Participant Flow}
  \begin{figure}
    \centering
    \includegraphics[height=0.8\textheight]{../04-analysis/outputs/figures/strobe_flow.png}
    \caption{STROBE Flow Diagram}
  \end{figure}
\end{frame}

\begin{frame}{Results: Population Characteristics}
  \textbf{Comparison: Periodontitis vs. Healthy}
  \begin{itemize}
    \item \textbf{Demographics}: Periodontitis group was older (53.8 vs 44.7 y) and had lower SES.
    \item \textbf{Behaviors}: Higher smoking rates (20.1\% vs 9.1\%).
    \item \textbf{Unadjusted Dietary Intake}:
      \begin{table}
      \centering
      \small
      \begin{tabular}{lcc}
        \toprule
        \textbf{Nutrient} & \textbf{No/Mild Perio} & \textbf{Mod/Severe Perio} \\
        \midrule
        Fiber (g) & 19.0 & 16.9 \\
        Vitamin C (mg) & 82.9 & 75.1 \\
        Calcium (mg) & 1045 & 928 \\
        \bottomrule
      \end{tabular}
      \end{table}
      \textit{All differences p < 0.001 (unadjusted).}
  \end{itemize}
\end{frame}

\begin{frame}{Results: Multivariable Regression}
  \textbf{Association with Moderate/Severe Periodontitis (Fully Adjusted)}
  \begin{table}
    \centering
    \begin{tabular}{lccc}
      \toprule
      \textbf{Nutrient} & \textbf{OR (95\% CI)} & \textbf{P-value} & \textbf{Interpretation} \\
      \midrule
      \textbf{Dietary Fiber} & \textbf{0.99 (0.99-0.99)} & \textbf{<0.001} & \textbf{Protective} \\
      Total Sugars & 1.00 (1.00-1.00) & <0.001 & Null \\
      Vitamin C & 1.00 (1.00-1.00) & <0.001 & Null \\
      Calcium & 1.00 (1.00-1.00) & <0.001 & Null \\
      \bottomrule
    \end{tabular}
  \end{table}
  \vspace{1em}
  \small{Note: OR is per unit change (1g or 1mg). Fiber OR of 0.99 implies $\sim$1\% risk reduction per gram.}
\end{frame}

\begin{frame}{Discussion}
  \begin{itemize}
    \item \textbf{Fiber}: Consistent with literature. 
      \begin{itemize}
        \item Mechanism: Prebiotic effect, glycemic control, reduced systemic inflammation.
        \item A 10g increase in daily fiber could reduce risk by $\sim$10\%.
      \end{itemize}
    \item \textbf{Sugars}: No independent association for \textit{Total Sugars}.
      \begin{itemize}
        \item Distinction between intrinsic (fruit) vs. added sugars is key.
      \end{itemize}
    \item \textbf{Micronutrients}: No independent effect after adjustment.
      \begin{itemize}
        \item Effects likely confounded by overall diet quality and smoking.
      \end{itemize}
  \end{itemize}
\end{frame}

\begin{frame}{Strengths and Limitations}
  \begin{columns}
    \begin{column}{0.5\textwidth}
      \textbf{Strengths}
      \begin{itemize}
        \item Large, nationally representative sample.
        \item Gold-standard periodontal exams.
        \item Comprehensive covariate adjustment.
      \end{itemize}
    \end{column}
    \begin{column}{0.5\textwidth}
      \textbf{Limitations}
      \begin{itemize}
        \item Cross-sectional (no causality).
        \item Single 24h recall (measurement error).
        \item Residual confounding (genetics, microbiome).
      \end{itemize}
    \end{column}
  \end{columns}
\end{frame}

\begin{frame}{Conclusion}
  \begin{block}{Take-Home Message}
    \textbf{Dietary fiber intake is independently and inversely associated with periodontitis in U.S. adults.}
  \end{block}
  \vspace{1em}
  \begin{itemize}
    \item Increasing fiber intake may be a simple, effective adjunct strategy for periodontal health.
    \item Future research should focus on longitudinal designs and "added sugars".
  \end{itemize}
\end{frame}

\begin{frame}
  \centering
  \Huge{Thank You}
  \vspace{1em}
  \normalsize
  \\
  Questions?
\end{frame}

\end{document}
